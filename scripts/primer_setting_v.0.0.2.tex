% Options for packages loaded elsewhere
\PassOptionsToPackage{unicode}{hyperref}
\PassOptionsToPackage{hyphens}{url}
%
\documentclass[
]{article}
\usepackage{amsmath,amssymb}
\usepackage{iftex}
\ifPDFTeX
  \usepackage[T1]{fontenc}
  \usepackage[utf8]{inputenc}
  \usepackage{textcomp} % provide euro and other symbols
\else % if luatex or xetex
  \usepackage{unicode-math} % this also loads fontspec
  \defaultfontfeatures{Scale=MatchLowercase}
  \defaultfontfeatures[\rmfamily]{Ligatures=TeX,Scale=1}
\fi
\usepackage{lmodern}
\ifPDFTeX\else
  % xetex/luatex font selection
\fi
% Use upquote if available, for straight quotes in verbatim environments
\IfFileExists{upquote.sty}{\usepackage{upquote}}{}
\IfFileExists{microtype.sty}{% use microtype if available
  \usepackage[]{microtype}
  \UseMicrotypeSet[protrusion]{basicmath} % disable protrusion for tt fonts
}{}
\makeatletter
\@ifundefined{KOMAClassName}{% if non-KOMA class
  \IfFileExists{parskip.sty}{%
    \usepackage{parskip}
  }{% else
    \setlength{\parindent}{0pt}
    \setlength{\parskip}{6pt plus 2pt minus 1pt}}
}{% if KOMA class
  \KOMAoptions{parskip=half}}
\makeatother
\usepackage{xcolor}
\usepackage[margin=1in]{geometry}
\usepackage{color}
\usepackage{fancyvrb}
\newcommand{\VerbBar}{|}
\newcommand{\VERB}{\Verb[commandchars=\\\{\}]}
\DefineVerbatimEnvironment{Highlighting}{Verbatim}{commandchars=\\\{\}}
% Add ',fontsize=\small' for more characters per line
\usepackage{framed}
\definecolor{shadecolor}{RGB}{248,248,248}
\newenvironment{Shaded}{\begin{snugshade}}{\end{snugshade}}
\newcommand{\AlertTok}[1]{\textcolor[rgb]{0.94,0.16,0.16}{#1}}
\newcommand{\AnnotationTok}[1]{\textcolor[rgb]{0.56,0.35,0.01}{\textbf{\textit{#1}}}}
\newcommand{\AttributeTok}[1]{\textcolor[rgb]{0.13,0.29,0.53}{#1}}
\newcommand{\BaseNTok}[1]{\textcolor[rgb]{0.00,0.00,0.81}{#1}}
\newcommand{\BuiltInTok}[1]{#1}
\newcommand{\CharTok}[1]{\textcolor[rgb]{0.31,0.60,0.02}{#1}}
\newcommand{\CommentTok}[1]{\textcolor[rgb]{0.56,0.35,0.01}{\textit{#1}}}
\newcommand{\CommentVarTok}[1]{\textcolor[rgb]{0.56,0.35,0.01}{\textbf{\textit{#1}}}}
\newcommand{\ConstantTok}[1]{\textcolor[rgb]{0.56,0.35,0.01}{#1}}
\newcommand{\ControlFlowTok}[1]{\textcolor[rgb]{0.13,0.29,0.53}{\textbf{#1}}}
\newcommand{\DataTypeTok}[1]{\textcolor[rgb]{0.13,0.29,0.53}{#1}}
\newcommand{\DecValTok}[1]{\textcolor[rgb]{0.00,0.00,0.81}{#1}}
\newcommand{\DocumentationTok}[1]{\textcolor[rgb]{0.56,0.35,0.01}{\textbf{\textit{#1}}}}
\newcommand{\ErrorTok}[1]{\textcolor[rgb]{0.64,0.00,0.00}{\textbf{#1}}}
\newcommand{\ExtensionTok}[1]{#1}
\newcommand{\FloatTok}[1]{\textcolor[rgb]{0.00,0.00,0.81}{#1}}
\newcommand{\FunctionTok}[1]{\textcolor[rgb]{0.13,0.29,0.53}{\textbf{#1}}}
\newcommand{\ImportTok}[1]{#1}
\newcommand{\InformationTok}[1]{\textcolor[rgb]{0.56,0.35,0.01}{\textbf{\textit{#1}}}}
\newcommand{\KeywordTok}[1]{\textcolor[rgb]{0.13,0.29,0.53}{\textbf{#1}}}
\newcommand{\NormalTok}[1]{#1}
\newcommand{\OperatorTok}[1]{\textcolor[rgb]{0.81,0.36,0.00}{\textbf{#1}}}
\newcommand{\OtherTok}[1]{\textcolor[rgb]{0.56,0.35,0.01}{#1}}
\newcommand{\PreprocessorTok}[1]{\textcolor[rgb]{0.56,0.35,0.01}{\textit{#1}}}
\newcommand{\RegionMarkerTok}[1]{#1}
\newcommand{\SpecialCharTok}[1]{\textcolor[rgb]{0.81,0.36,0.00}{\textbf{#1}}}
\newcommand{\SpecialStringTok}[1]{\textcolor[rgb]{0.31,0.60,0.02}{#1}}
\newcommand{\StringTok}[1]{\textcolor[rgb]{0.31,0.60,0.02}{#1}}
\newcommand{\VariableTok}[1]{\textcolor[rgb]{0.00,0.00,0.00}{#1}}
\newcommand{\VerbatimStringTok}[1]{\textcolor[rgb]{0.31,0.60,0.02}{#1}}
\newcommand{\WarningTok}[1]{\textcolor[rgb]{0.56,0.35,0.01}{\textbf{\textit{#1}}}}
\usepackage{graphicx}
\makeatletter
\def\maxwidth{\ifdim\Gin@nat@width>\linewidth\linewidth\else\Gin@nat@width\fi}
\def\maxheight{\ifdim\Gin@nat@height>\textheight\textheight\else\Gin@nat@height\fi}
\makeatother
% Scale images if necessary, so that they will not overflow the page
% margins by default, and it is still possible to overwrite the defaults
% using explicit options in \includegraphics[width, height, ...]{}
\setkeys{Gin}{width=\maxwidth,height=\maxheight,keepaspectratio}
% Set default figure placement to htbp
\makeatletter
\def\fps@figure{htbp}
\makeatother
\usepackage{soul}
\setlength{\emergencystretch}{3em} % prevent overfull lines
\providecommand{\tightlist}{%
  \setlength{\itemsep}{0pt}\setlength{\parskip}{0pt}}
\setcounter{secnumdepth}{-\maxdimen} % remove section numbering
\ifLuaTeX
  \usepackage{selnolig}  % disable illegal ligatures
\fi
\IfFileExists{bookmark.sty}{\usepackage{bookmark}}{\usepackage{hyperref}}
\IfFileExists{xurl.sty}{\usepackage{xurl}}{} % add URL line breaks if available
\urlstyle{same}
\hypersetup{
  pdftitle={primer\_selection\_v.0.0.2},
  pdfauthor={Hongxi},
  hidelinks,
  pdfcreator={LaTeX via pandoc}}

\title{primer\_selection\_v.0.0.2}
\author{Hongxi}
\date{2023-11-28}

\begin{document}
\maketitle

\hypertarget{ux5f15ux7269ux8bbeux8ba1ux6d41ux7a0b}{%
\section{引物设计流程}\label{ux5f15ux7269ux8bbeux8ba1ux6d41ux7a0b}}

v.0.0.2\\
更新:\\
1. 更新了Tm值的算法,现在使用的是改良后的最邻近法,其公式为:\\

\(\Delta H = \frac{\Delta H}{\Delta S+A+Rln(\frac{C}{4})}-273.15+mlog[Na^+]\)\\
其中,\(Delta H\)和\(Delta S\)通过查表并累加可得。A为起始螺旋起始所需的一个常数,R为气体常数,C为DNA单链的浓度,m则是钠离子浓度的对数的系数。该公式的各个参数可能需要后续实验进行校准。\\
2. 由于目前的Tm值算法与GC含量高度相关,GC含量的筛选条件已被移除\\
3.
计算Tm值时会顺便返回引物与互补链结合的自由能,但该信息未在最终结果中被提取出来

后续计划:\\
1. 计算扩增子(50-150bp)的Tm值,有望用于SYBR相关的熔解曲线预测\\
2. 增加对于重复序列的检测\\
3. 增加基于序列相似性或自由能的打分系统,有望对于结果进行排序\\
3.1. 基于计划3,完善序列比对功能\\
4. 二级结构预测\\
5. 优化比对函数,更快的运行速度 6. 输入序列的保守区域预测 7.
引物的种间特异性检测

\hypertarget{ux5b9aux4e49ux6240ux9700ux51fdux6570}{%
\subsection{定义所需函数}\label{ux5b9aux4e49ux6240ux9700ux51fdux6570}}

在流程开始前定义流程所需的函数,其包括:\\
1.
\texttt{reverseDNA}:反向互补序列函数。输入\textbf{DNA序列}的反向DNA序列\\
2. \texttt{align\_all}:
序列比对函数。输入A、B两条序列,将返回A序列在B序列(以及B序列的反向互补序列)中出现的总次数。\\
3. \texttt{GC\_contain}:
计算GC含量的函数。返回输入序列的GC百分比含量。\\
4.
\texttt{get\_all\_primers}:批量获取引物序列。从输入的序列中获取所有在某个长度范围内的子序列(以及其所在位置)并放入一个数据框后返回。需要输入的参数有最小值(\texttt{minsize})、最大值(\texttt{maxsize})、DNA长度(\texttt{DNAsize})、DNA序列(\texttt{DNAseq})。\\
5.
\texttt{get\_Tm\_G}:根据最邻近法计算目标序列的Tm值与自由能。默认单价阳离子浓度为0.05M,模板浓度为25uM,温度为37℃。修改默认温度可能导致结果不稳定。

\begin{Shaded}
\begin{Highlighting}[]
\CommentTok{\#清理参数并记录时间}
\FunctionTok{rm}\NormalTok{(}\AttributeTok{list =} \FunctionTok{ls}\NormalTok{())}
\NormalTok{start\_time }\OtherTok{\textless{}{-}} \FunctionTok{Sys.time}\NormalTok{()}

\CommentTok{\#配置函数}
\DocumentationTok{\#\#获取反向DNA序列(5\textquotesingle{}{-}3\textquotesingle{})}
\NormalTok{reverseDNA }\OtherTok{\textless{}{-}} \ControlFlowTok{function}\NormalTok{(DNA) \{}
\NormalTok{  q1 }\OtherTok{\textless{}{-}} \FunctionTok{c}\NormalTok{(}\StringTok{"A"}\NormalTok{,}\StringTok{"G"}\NormalTok{,}\StringTok{"C"}\NormalTok{,}\StringTok{"T"}\NormalTok{)}
\NormalTok{  q2 }\OtherTok{\textless{}{-}} \FunctionTok{c}\NormalTok{(}\StringTok{"T"}\NormalTok{,}\StringTok{"C"}\NormalTok{,}\StringTok{"G"}\NormalTok{,}\StringTok{"A"}\NormalTok{)}
  
  \FunctionTok{names}\NormalTok{(q2)}\OtherTok{=}\NormalTok{q1}
  
\NormalTok{  F\_seq }\OtherTok{\textless{}{-}}\NormalTok{ DNA}
\NormalTok{  R\_seq }\OtherTok{\textless{}{-}} \FunctionTok{paste}\NormalTok{(}\FunctionTok{rev}\NormalTok{(q2[}\FunctionTok{unlist}\NormalTok{(}\FunctionTok{strsplit}\NormalTok{(F\_seq,}\StringTok{""}\NormalTok{))]),}\AttributeTok{collapse =} \StringTok{""}\NormalTok{)}
  
  \FunctionTok{return}\NormalTok{(R\_seq)}
\NormalTok{\}}

\DocumentationTok{\#\#序列比对函数:返回相同序列的个数(包括反向序列上的)}
\NormalTok{align\_all }\OtherTok{\textless{}{-}} \ControlFlowTok{function}\NormalTok{(query,subject) \{}
  
\NormalTok{  c\_query }\OtherTok{\textless{}{-}} \FunctionTok{reverseDNA}\NormalTok{(}\AttributeTok{DNA=}\NormalTok{query)}
  
\NormalTok{  F\_matches }\OtherTok{\textless{}{-}} \FunctionTok{str\_count}\NormalTok{(subject,query)}
\NormalTok{  R\_matches }\OtherTok{\textless{}{-}} \FunctionTok{str\_count}\NormalTok{(subject,c\_query)}
  
\NormalTok{  all\_matches }\OtherTok{\textless{}{-}}\NormalTok{ F\_matches }\SpecialCharTok{+}\NormalTok{ R\_matches}
  \FunctionTok{return}\NormalTok{(all\_matches)}
\NormalTok{\}}

\DocumentationTok{\#\#GC含量函数}
\NormalTok{GC\_contain }\OtherTok{\textless{}{-}} \ControlFlowTok{function}\NormalTok{(DNA) \{}
\NormalTok{  GC }\OtherTok{\textless{}{-}}\NormalTok{ (}\FunctionTok{str\_count}\NormalTok{(DNA, }\StringTok{"G"}\NormalTok{)}\SpecialCharTok{+}\FunctionTok{str\_count}\NormalTok{(DNA, }\StringTok{"C"}\NormalTok{))}\SpecialCharTok{/}\FunctionTok{nchar}\NormalTok{(DNA)}\SpecialCharTok{*}\DecValTok{100}
\NormalTok{  GC }\OtherTok{\textless{}{-}} \FunctionTok{round}\NormalTok{(GC,}\DecValTok{2}\NormalTok{)}
  \FunctionTok{return}\NormalTok{(GC)}
\NormalTok{\}}

\DocumentationTok{\#\#计算Tm值与自由能并合为一个向量输出}
\NormalTok{get\_Tm\_G }\OtherTok{\textless{}{-}} \ControlFlowTok{function}\NormalTok{(seq,value\_table,}\AttributeTok{C\_Na=}\FloatTok{0.05}\NormalTok{,}\AttributeTok{C\_seq=}\FloatTok{2.5e{-}7}\NormalTok{,}\AttributeTok{tem=}\DecValTok{37}\NormalTok{,}\AttributeTok{A=}\SpecialCharTok{{-}}\FloatTok{0.0108}\NormalTok{,}\AttributeTok{R=}\FloatTok{0.001987}\NormalTok{) \{}
\NormalTok{  get\_pairs }\OtherTok{\textless{}{-}} \ControlFlowTok{function}\NormalTok{(sequence) \{}
\NormalTok{    pairs }\OtherTok{\textless{}{-}} \FunctionTok{sapply}\NormalTok{(}\DecValTok{1}\SpecialCharTok{:}\NormalTok{(}\FunctionTok{nchar}\NormalTok{(sequence)}\SpecialCharTok{{-}}\DecValTok{1}\NormalTok{),}\ControlFlowTok{function}\NormalTok{(i)\{}
      \FunctionTok{substr}\NormalTok{(sequence,i,i}\SpecialCharTok{+}\DecValTok{1}\NormalTok{)}
\NormalTok{    \})}
    \FunctionTok{return}\NormalTok{(pairs)}
\NormalTok{  \}}
  
\NormalTok{  adjacent\_pairs }\OtherTok{\textless{}{-}} \FunctionTok{get\_pairs}\NormalTok{(seq)}
  
\NormalTok{  H\_values }\OtherTok{\textless{}{-}} \FunctionTok{sum}\NormalTok{(value\_table}\SpecialCharTok{$}\NormalTok{H[}\FunctionTok{match}\NormalTok{(adjacent\_pairs,value\_table}\SpecialCharTok{$}\NormalTok{seq)])}
\NormalTok{  S\_values }\OtherTok{\textless{}{-}} \FunctionTok{sum}\NormalTok{(value\_table}\SpecialCharTok{$}\NormalTok{S[}\FunctionTok{match}\NormalTok{(adjacent\_pairs,value\_table}\SpecialCharTok{$}\NormalTok{seq)])}
  
\NormalTok{  H\_init }\OtherTok{\textless{}{-}} \FunctionTok{sum}\NormalTok{(value\_table}\SpecialCharTok{$}\NormalTok{init\_H[}\FunctionTok{match}\NormalTok{(}\FunctionTok{c}\NormalTok{(adjacent\_pairs[}\DecValTok{1}\NormalTok{],adjacent\_pairs[}\FunctionTok{length}\NormalTok{(adjacent\_pairs)]),value\_table}\SpecialCharTok{$}\NormalTok{seq)])}
\NormalTok{  S\_init }\OtherTok{\textless{}{-}} \FunctionTok{sum}\NormalTok{(value\_table}\SpecialCharTok{$}\NormalTok{init\_S[}\FunctionTok{match}\NormalTok{(}\FunctionTok{c}\NormalTok{(adjacent\_pairs[}\DecValTok{1}\NormalTok{],adjacent\_pairs[}\FunctionTok{length}\NormalTok{(adjacent\_pairs)]),value\_table}\SpecialCharTok{$}\NormalTok{seq)])}
  
\NormalTok{  G }\OtherTok{\textless{}{-}}\NormalTok{ (H\_values}\SpecialCharTok{+}\NormalTok{H\_init) }\SpecialCharTok{{-}}\NormalTok{ (tem}\FloatTok{+273.15}\NormalTok{)}\SpecialCharTok{*}\NormalTok{(S\_values}\SpecialCharTok{+}\NormalTok{S\_init)}
\NormalTok{  Tm }\OtherTok{\textless{}{-}}\NormalTok{ H\_values}\SpecialCharTok{/}\NormalTok{(S\_values}\SpecialCharTok{+}\NormalTok{A}\SpecialCharTok{+}\NormalTok{R}\SpecialCharTok{*}\FunctionTok{log}\NormalTok{(C\_seq}\SpecialCharTok{/}\DecValTok{4}\NormalTok{))}\SpecialCharTok{{-}}\FloatTok{273.15}\SpecialCharTok{+}\NormalTok{(}\FloatTok{11.4}\SpecialCharTok{*}\FunctionTok{log10}\NormalTok{(C\_Na))}
  
  \FunctionTok{return}\NormalTok{(}\FunctionTok{c}\NormalTok{(Tm,G))}
\NormalTok{\}}

\DocumentationTok{\#\#获取字符串中特定长度的子集并标注位置}
\NormalTok{get\_all\_primers }\OtherTok{\textless{}{-}} \ControlFlowTok{function}\NormalTok{(minsize, maxsize, DNAsize, DNAseq) \{}

\NormalTok{  get\_primer\_positions }\OtherTok{\textless{}{-}} \ControlFlowTok{function}\NormalTok{(minsize, maxsize, DNAsize) \{}
    
\NormalTok{    primer\_size\_range }\OtherTok{\textless{}{-}}\NormalTok{ minsize}\SpecialCharTok{:}\NormalTok{maxsize}
    
\NormalTok{    get\_position }\OtherTok{\textless{}{-}} \ControlFlowTok{function}\NormalTok{(DNAsize,primer\_size) \{}
\NormalTok{    startpoint }\OtherTok{\textless{}{-}} \DecValTok{1}\SpecialCharTok{:}\NormalTok{(DNAsize }\SpecialCharTok{{-}}\NormalTok{ primer\_size }\SpecialCharTok{+} \DecValTok{1}\NormalTok{)}
\NormalTok{    endpoint }\OtherTok{\textless{}{-}}\NormalTok{ primer\_size}\SpecialCharTok{:}\NormalTok{DNAsize}
    
\NormalTok{    temdata }\OtherTok{\textless{}{-}} \FunctionTok{data.frame}\NormalTok{(}\AttributeTok{start =}\NormalTok{ startpoint, }\AttributeTok{end =}\NormalTok{ endpoint)}
    
    \FunctionTok{return}\NormalTok{(temdata)}
\NormalTok{    \}}
    
\NormalTok{    position\_list }\OtherTok{\textless{}{-}} \FunctionTok{mapply}\NormalTok{(get\_position, }
                            \AttributeTok{DNAsize =}\NormalTok{ DNAsize, }
                            \AttributeTok{primer\_size =}\NormalTok{ primer\_size\_range)}
    
\NormalTok{    position\_map }\OtherTok{\textless{}{-}} \FunctionTok{lapply}\NormalTok{(}\FunctionTok{data.frame}\NormalTok{(position\_list), as.list)}
    
\NormalTok{    primer\_all\_position }\OtherTok{\textless{}{-}} \FunctionTok{do.call}\NormalTok{(bind\_rows, position\_map)}
\NormalTok{    primer\_all\_position}\SpecialCharTok{$}\NormalTok{accession }\OtherTok{\textless{}{-}} \DecValTok{1}\SpecialCharTok{:}\FunctionTok{length}\NormalTok{(primer\_all\_position}\SpecialCharTok{$}\NormalTok{start)}
    
    \FunctionTok{return}\NormalTok{(primer\_all\_position)}
\NormalTok{  \}}

\NormalTok{  primer\_positions }\OtherTok{\textless{}{-}} \FunctionTok{get\_primer\_positions}\NormalTok{(}\AttributeTok{minsize =}\NormalTok{ minsize,}
                                           \AttributeTok{maxsize =}\NormalTok{ maxsize,}
                                           \AttributeTok{DNAsize =}\NormalTok{ DNAsize)}
  
\NormalTok{  primer\_positions}\SpecialCharTok{$}\NormalTok{seq }\OtherTok{\textless{}{-}} \FunctionTok{str\_sub}\NormalTok{(DNAseq, }\AttributeTok{start =}\NormalTok{ primer\_positions}\SpecialCharTok{$}\NormalTok{start, }\AttributeTok{end =}\NormalTok{ primer\_positions}\SpecialCharTok{$}\NormalTok{end)}

  \FunctionTok{return}\NormalTok{(primer\_positions)}
\NormalTok{\}}
\end{Highlighting}
\end{Shaded}

\hypertarget{ux52a0ux8f7dux6240ux9700ux8981ux7684ux5305}{%
\subsection{加载所需要的包}\label{ux52a0ux8f7dux6240ux9700ux8981ux7684ux5305}}

该流程需要使用以下常用处理字符串的包:\\
1. \texttt{stringr}\\
2. \texttt{dplyr}\\
3. \texttt{tidyr}

运行以下代码加载包

\begin{Shaded}
\begin{Highlighting}[]
\CommentTok{\#加载包}
\FunctionTok{library}\NormalTok{(stringr)}
\FunctionTok{library}\NormalTok{(dplyr)}
\FunctionTok{library}\NormalTok{(tidyr)}
\end{Highlighting}
\end{Shaded}

\hypertarget{ux5f00ux59cbux5206ux6790}{%
\subsection{开始分析}\label{ux5f00ux59cbux5206ux6790}}

\begin{Shaded}
\begin{Highlighting}[]
\CommentTok{\#加载数据并初步处理}
\FunctionTok{setwd}\NormalTok{(}\StringTok{"C:/Users/Hongxi/Documents/R\_project"}\NormalTok{)}

\NormalTok{input\_path }\OtherTok{\textless{}{-}} \StringTok{"./sequences/18S\_rRNA\_N1.txt"}

\NormalTok{NN\_values }\OtherTok{\textless{}{-}} \FunctionTok{read.csv}\NormalTok{(}\StringTok{"./scripts/NN\_values.csv"}\NormalTok{)}
\NormalTok{F\_DNA }\OtherTok{\textless{}{-}} \FunctionTok{paste}\NormalTok{(}\FunctionTok{readLines}\NormalTok{(input\_path), }\AttributeTok{collapse =} \StringTok{""}\NormalTok{)}
\NormalTok{R\_DNA }\OtherTok{\textless{}{-}} \FunctionTok{reverseDNA}\NormalTok{(}\AttributeTok{DNA =}\NormalTok{ F\_DNA)}
\end{Highlighting}
\end{Shaded}

\begin{Shaded}
\begin{Highlighting}[]
\CommentTok{\#目标序列基础分析}
\NormalTok{target\_size }\OtherTok{\textless{}{-}} \FunctionTok{nchar}\NormalTok{(F\_DNA)}

\NormalTok{target\_GC }\OtherTok{\textless{}{-}} \FunctionTok{GC\_contain}\NormalTok{(F\_DNA)}
\end{Highlighting}
\end{Shaded}

\hypertarget{ux5355ux6761ux5f15ux7269ux7684ux9884ux7b5bux9009}{%
\subsection{单条引物的预筛选}\label{ux5355ux6761ux5f15ux7269ux7684ux9884ux7b5bux9009}}

根据引物大小,使用本流程预先设定的函数\texttt{get\_all\_primers}选出所有长度范围内的子序列以减少运算量。

\begin{Shaded}
\begin{Highlighting}[]
\CommentTok{\#输入引物大小限制}

\NormalTok{primer\_size\_min }\OtherTok{\textless{}{-}} \DecValTok{18}
\NormalTok{primer\_size\_max }\OtherTok{\textless{}{-}} \DecValTok{23}

\CommentTok{\#获取潜在引物序列}
\NormalTok{F\_primer\_all }\OtherTok{\textless{}{-}} \FunctionTok{get\_all\_primers}\NormalTok{(}\AttributeTok{minsize =}\NormalTok{ primer\_size\_min,}
                                \AttributeTok{maxsize =}\NormalTok{ primer\_size\_max,}
                                \AttributeTok{DNAsize =}\NormalTok{ target\_size,}
                                \AttributeTok{DNAseq =}\NormalTok{ F\_DNA)}
\NormalTok{R\_primer\_all }\OtherTok{\textless{}{-}} \FunctionTok{get\_all\_primers}\NormalTok{(}\AttributeTok{minsize =}\NormalTok{ primer\_size\_min,}
                                \AttributeTok{maxsize =}\NormalTok{ primer\_size\_max,}
                                \AttributeTok{DNAsize =}\NormalTok{ target\_size,}
                                \AttributeTok{DNAseq =}\NormalTok{ R\_DNA)}
\end{Highlighting}
\end{Shaded}

\hypertarget{ux5355ux6761ux5f15ux7269ux7684ux7b5bux9009}{%
\subsection{单条引物的筛选}\label{ux5355ux6761ux5f15ux7269ux7684ux7b5bux9009}}

这里使用的限制条件有:\\
1. 引物大小\\
2. \st{CG含量}\\
3. 引物Tm:\\
引物的Tm值的计算方式为改良后的最邻近法,其中参考了单价阳离子浓度和模板浓度的参数
4. 引物自身3端配对\\
5. 引物与靶标的3端配对

其中,引物自身配对的限制是为了尽量避免发卡结构和引物二聚体的产生,与靶标的3端限制是为了避免与靶标的非特异性扩增。但是这种方法无法准确排除发卡结构和引物二聚体,因此外部软件的介入可能是必要的。

调整相关参数(GC含量、Tm范围、引物自身3端配对上限(含)、引物与靶标配对上限(含))后运行以下代码,可得两个分别包含正反向可用单条引物的数据框

\begin{Shaded}
\begin{Highlighting}[]
\CommentTok{\#单引物筛选}
\DocumentationTok{\#\#计算CG含量(此处应该可以优化)}
\NormalTok{F\_primer\_all}\SpecialCharTok{$}\NormalTok{GC }\OtherTok{\textless{}{-}} \FunctionTok{sapply}\NormalTok{(}\FunctionTok{as.list}\NormalTok{(F\_primer\_all}\SpecialCharTok{$}\NormalTok{seq), GC\_contain)}
\NormalTok{R\_primer\_all}\SpecialCharTok{$}\NormalTok{GC }\OtherTok{\textless{}{-}} \FunctionTok{sapply}\NormalTok{(}\FunctionTok{as.list}\NormalTok{(R\_primer\_all}\SpecialCharTok{$}\NormalTok{seq), GC\_contain)}

\DocumentationTok{\#\#计算Tm和自由能}
\NormalTok{F\_Tm\_G }\OtherTok{\textless{}{-}} \FunctionTok{sapply}\NormalTok{(}\FunctionTok{as.list}\NormalTok{(F\_primer\_all}\SpecialCharTok{$}\NormalTok{seq),get\_Tm\_G,}\AttributeTok{value\_table =}\NormalTok{ NN\_values)}
\NormalTok{R\_Tm\_G }\OtherTok{\textless{}{-}} \FunctionTok{sapply}\NormalTok{(}\FunctionTok{as.list}\NormalTok{(R\_primer\_all}\SpecialCharTok{$}\NormalTok{seq),get\_Tm\_G,}\AttributeTok{value\_table =}\NormalTok{ NN\_values)}

\NormalTok{F\_primer\_all}\SpecialCharTok{$}\NormalTok{Tm }\OtherTok{\textless{}{-}}\NormalTok{ F\_Tm\_G[}\DecValTok{1}\NormalTok{,]}
\NormalTok{F\_primer\_all}\SpecialCharTok{$}\NormalTok{G }\OtherTok{\textless{}{-}}\NormalTok{ F\_Tm\_G[}\DecValTok{2}\NormalTok{,]}

\NormalTok{R\_primer\_all}\SpecialCharTok{$}\NormalTok{Tm }\OtherTok{\textless{}{-}}\NormalTok{ R\_Tm\_G[}\DecValTok{1}\NormalTok{,]}
\NormalTok{R\_primer\_all}\SpecialCharTok{$}\NormalTok{G }\OtherTok{\textless{}{-}}\NormalTok{ R\_Tm\_G[}\DecValTok{2}\NormalTok{,]}

\FunctionTok{rm}\NormalTok{(F\_Tm\_G,R\_Tm\_G)}

\DocumentationTok{\#\#设定Tm阈值 (后期tm阈值应与GC含量协同)}
\NormalTok{Tm\_min }\OtherTok{\textless{}{-}} \DecValTok{58}
\NormalTok{Tm\_max }\OtherTok{\textless{}{-}} \DecValTok{63}

\DocumentationTok{\#\#根据Tm值筛选}
\NormalTok{F\_primer }\OtherTok{\textless{}{-}} \FunctionTok{subset}\NormalTok{(F\_primer\_all, }\AttributeTok{subset =}\NormalTok{ Tm }\SpecialCharTok{\textgreater{}=}\NormalTok{ Tm\_min }\SpecialCharTok{\&}\NormalTok{ Tm }\SpecialCharTok{\textless{}=}\NormalTok{ Tm\_max)}
\NormalTok{R\_primer }\OtherTok{\textless{}{-}} \FunctionTok{subset}\NormalTok{(R\_primer\_all, }\AttributeTok{subset =}\NormalTok{ Tm }\SpecialCharTok{\textgreater{}=}\NormalTok{ Tm\_min }\SpecialCharTok{\&}\NormalTok{ Tm }\SpecialCharTok{\textless{}=}\NormalTok{ Tm\_max)}
\end{Highlighting}
\end{Shaded}

\begin{Shaded}
\begin{Highlighting}[]
\DocumentationTok{\#\#输入引物3端配对数上限}
\NormalTok{self\_match\_max }\OtherTok{\textless{}{-}} \DecValTok{5}
\DocumentationTok{\#\#引物3端自身配对筛选}
\NormalTok{F\_primer}\SpecialCharTok{$}\NormalTok{reversed\_tail }\OtherTok{\textless{}{-}} \FunctionTok{str\_sub}\NormalTok{(}\FunctionTok{sapply}\NormalTok{(}\FunctionTok{as.list}\NormalTok{(F\_primer}\SpecialCharTok{$}\NormalTok{seq),reverseDNA), }\AttributeTok{end =}\NormalTok{ self\_match\_max)}
\NormalTok{F\_primer}\SpecialCharTok{$}\NormalTok{self\_tail\_match }\OtherTok{\textless{}{-}} \FunctionTok{str\_count}\NormalTok{(F\_primer}\SpecialCharTok{$}\NormalTok{seq,F\_primer}\SpecialCharTok{$}\NormalTok{reversed\_tail)}


\NormalTok{R\_primer}\SpecialCharTok{$}\NormalTok{reversed\_tail }\OtherTok{\textless{}{-}} \FunctionTok{str\_sub}\NormalTok{(}\FunctionTok{sapply}\NormalTok{(}\FunctionTok{as.list}\NormalTok{(R\_primer}\SpecialCharTok{$}\NormalTok{seq),reverseDNA), }\AttributeTok{end =}\NormalTok{ self\_match\_max)}
\NormalTok{R\_primer}\SpecialCharTok{$}\NormalTok{self\_tail\_match }\OtherTok{\textless{}{-}} \FunctionTok{str\_count}\NormalTok{(R\_primer}\SpecialCharTok{$}\NormalTok{seq,R\_primer}\SpecialCharTok{$}\NormalTok{reversed\_tail)}

\NormalTok{F\_primer }\OtherTok{\textless{}{-}} \FunctionTok{subset}\NormalTok{(F\_primer,}
                   \AttributeTok{subset =}\NormalTok{ self\_tail\_match }\SpecialCharTok{==} \DecValTok{0}\NormalTok{,}
                   \AttributeTok{select =} \FunctionTok{c}\NormalTok{(}\StringTok{"accession"}\NormalTok{,}\StringTok{"seq"}\NormalTok{,}\StringTok{"start"}\NormalTok{,}\StringTok{"end"}\NormalTok{,}\StringTok{"GC"}\NormalTok{,}\StringTok{"Tm"}\NormalTok{))}
\NormalTok{R\_primer }\OtherTok{\textless{}{-}} \FunctionTok{subset}\NormalTok{(R\_primer,}
                   \AttributeTok{subset =}\NormalTok{ self\_tail\_match }\SpecialCharTok{==} \DecValTok{0}\NormalTok{,}
                   \AttributeTok{select =} \FunctionTok{c}\NormalTok{(}\StringTok{"accession"}\NormalTok{,}\StringTok{"seq"}\NormalTok{,}\StringTok{"start"}\NormalTok{,}\StringTok{"end"}\NormalTok{,}\StringTok{"GC"}\NormalTok{,}\StringTok{"Tm"}\NormalTok{))}


\DocumentationTok{\#\#输入引物3端与靶标配对数上限}
\NormalTok{target\_match\_max }\OtherTok{\textless{}{-}}\DecValTok{7}
\DocumentationTok{\#\#引物3端与靶标筛选}
\NormalTok{F\_primer}\SpecialCharTok{$}\NormalTok{tail }\OtherTok{\textless{}{-}} \FunctionTok{str\_sub}\NormalTok{(F\_primer}\SpecialCharTok{$}\NormalTok{seq, }\SpecialCharTok{{-}}\NormalTok{ target\_match\_max)}
\NormalTok{F\_primer}\SpecialCharTok{$}\NormalTok{tail\_match }\OtherTok{\textless{}{-}} \FunctionTok{mapply}\NormalTok{(align\_all, }\FunctionTok{as.list}\NormalTok{(F\_primer}\SpecialCharTok{$}\NormalTok{tail), }\FunctionTok{as.list}\NormalTok{(F\_primer}\SpecialCharTok{$}\NormalTok{seq))}

\NormalTok{R\_primer}\SpecialCharTok{$}\NormalTok{tail }\OtherTok{\textless{}{-}} \FunctionTok{str\_sub}\NormalTok{(R\_primer}\SpecialCharTok{$}\NormalTok{seq, }\SpecialCharTok{{-}}\NormalTok{ target\_match\_max)}
\NormalTok{R\_primer}\SpecialCharTok{$}\NormalTok{tail\_match }\OtherTok{\textless{}{-}} \FunctionTok{mapply}\NormalTok{(align\_all, }\FunctionTok{as.list}\NormalTok{(R\_primer}\SpecialCharTok{$}\NormalTok{tail), }\FunctionTok{as.list}\NormalTok{(R\_primer}\SpecialCharTok{$}\NormalTok{seq))}

\NormalTok{F\_primer }\OtherTok{\textless{}{-}} \FunctionTok{subset}\NormalTok{(F\_primer,}
                   \AttributeTok{subset =}\NormalTok{ tail\_match }\SpecialCharTok{==} \DecValTok{1}\NormalTok{,}
                   \AttributeTok{select =} \FunctionTok{c}\NormalTok{(}\StringTok{"accession"}\NormalTok{,}\StringTok{"seq"}\NormalTok{,}\StringTok{"start"}\NormalTok{,}\StringTok{"end"}\NormalTok{,}\StringTok{"GC"}\NormalTok{,}\StringTok{"Tm"}\NormalTok{))}
\FunctionTok{colnames}\NormalTok{(F\_primer) }\OtherTok{\textless{}{-}} \FunctionTok{c}\NormalTok{(}\StringTok{"F\_accession"}\NormalTok{, }\StringTok{"F\_seq"}\NormalTok{, }\StringTok{"F\_start"}\NormalTok{, }\StringTok{"F\_end"}\NormalTok{,}\StringTok{"F\_GC"}\NormalTok{,}\StringTok{"F\_Tm"}\NormalTok{)}

\NormalTok{R\_primer }\OtherTok{\textless{}{-}} \FunctionTok{subset}\NormalTok{(R\_primer,}
                   \AttributeTok{subset =}\NormalTok{ tail\_match }\SpecialCharTok{==} \DecValTok{1}\NormalTok{,}
                   \AttributeTok{select =} \FunctionTok{c}\NormalTok{(}\StringTok{"accession"}\NormalTok{,}\StringTok{"seq"}\NormalTok{,}\StringTok{"start"}\NormalTok{,}\StringTok{"end"}\NormalTok{,}\StringTok{"GC"}\NormalTok{,}\StringTok{"Tm"}\NormalTok{))}
\FunctionTok{colnames}\NormalTok{(R\_primer) }\OtherTok{\textless{}{-}} \FunctionTok{c}\NormalTok{(}\StringTok{"R\_accession"}\NormalTok{, }\StringTok{"R\_seq"}\NormalTok{, }\StringTok{"R\_start"}\NormalTok{, }\StringTok{"R\_end"}\NormalTok{,}\StringTok{"R\_GC"}\NormalTok{,}\StringTok{"R\_Tm"}\NormalTok{)}

\CommentTok{\#查看前20行引物所在的数据框}
\FunctionTok{print}\NormalTok{(F\_primer[}\DecValTok{1}\SpecialCharTok{:}\DecValTok{20}\NormalTok{,])}
\end{Highlighting}
\end{Shaded}

\begin{verbatim}
## # A tibble: 20 x 6
##    F_accession F_seq              F_start F_end  F_GC  F_Tm
##          <int> <chr>                <int> <int> <dbl> <dbl>
##  1         173 ACATGCCGACGGGCGCTG     173   190  72.2  58.6
##  2         174 CATGCCGACGGGCGCTGA     174   191  72.2  58.3
##  3         175 ATGCCGACGGGCGCTGAC     175   192  72.2  58.4
##  4         176 TGCCGACGGGCGCTGACC     176   193  77.8  60.6
##  5         177 GCCGACGGGCGCTGACCC     177   194  83.3  61.7
##  6         178 CCGACGGGCGCTGACCCC     178   195  83.3  61.2
##  7         179 CGACGGGCGCTGACCCCC     179   196  83.3  61.2
##  8         180 GACGGGCGCTGACCCCCT     180   197  77.8  59.7
##  9         181 ACGGGCGCTGACCCCCTT     181   198  72.2  58.9
## 10         182 CGGGCGCTGACCCCCTTC     182   199  77.8  58.6
## 11         183 GGGCGCTGACCCCCTTCG     183   200  77.8  58.6
## 12         184 GGCGCTGACCCCCTTCGC     184   201  77.8  59.2
## 13         185 GCGCTGACCCCCTTCGCG     185   202  77.8  59.4
## 14         186 CGCTGACCCCCTTCGCGG     186   203  77.8  58.8
## 15         187 GCTGACCCCCTTCGCGGG     187   204  77.8  58.6
## 16         188 CTGACCCCCTTCGCGGGG     188   205  77.8  58.1
## 17         192 CCCCCTTCGCGGGGGGGA     192   209  83.3  61.8
## 18         193 CCCCTTCGCGGGGGGGAT     193   210  77.8  59.4
## 19         194 CCCTTCGCGGGGGGGATG     194   211  77.8  58.2
## 20         195 CCTTCGCGGGGGGGATGC     195   212  77.8  58.8
\end{verbatim}

\begin{Shaded}
\begin{Highlighting}[]
\FunctionTok{print}\NormalTok{(R\_primer[}\DecValTok{1}\SpecialCharTok{:}\DecValTok{20}\NormalTok{,])}
\end{Highlighting}
\end{Shaded}

\begin{verbatim}
## # A tibble: 20 x 6
##    R_accession R_seq              R_start R_end  R_GC  R_Tm
##          <int> <chr>                <int> <int> <dbl> <dbl>
##  1          81 CTCAGCGCTCCGCCAGGG      81    98  77.8  58.6
##  2          82 TCAGCGCTCCGCCAGGGC      82    99  77.8  60.5
##  3          83 CAGCGCTCCGCCAGGGCC      83   100  83.3  61.9
##  4          86 CGCTCCGCCAGGGCCGTG      86   103  83.3  61.7
##  5          87 GCTCCGCCAGGGCCGTGG      87   104  83.3  61.6
##  6          88 CTCCGCCAGGGCCGTGGG      88   105  83.3  61.1
##  7          93 CCAGGGCCGTGGGCCGAC      93   110  83.3  61.3
##  8          94 CAGGGCCGTGGGCCGACC      94   111  83.3  61.3
##  9          95 AGGGCCGTGGGCCGACCC      95   112  83.3  62.5
## 10         108 GACCCCGGCGGGGCCGAT     108   125  83.3  62.5
## 11         109 ACCCCGGCGGGGCCGATC     109   126  83.3  62.5
## 12         112 CCGGCGGGGCCGATCCGA     112   129  83.3  62.3
## 13         113 CGGCGGGGCCGATCCGAG     113   130  83.3  60.9
## 14         114 GGCGGGGCCGATCCGAGG     114   131  83.3  60.8
## 15         115 GCGGGGCCGATCCGAGGG     115   132  83.3  60.8
## 16         116 CGGGGCCGATCCGAGGGC     116   133  83.3  60.8
## 17         117 GGGGCCGATCCGAGGGCC     117   134  83.3  60.7
## 18         118 GGGCCGATCCGAGGGCCT     118   135  77.8  59.3
## 19         286 CGGGTTACCCGCGCCTGC     286   303  77.8  59.4
## 20         287 GGGTTACCCGCGCCTGCC     287   304  77.8  59.3
\end{verbatim}

\begin{center}\rule{0.5\linewidth}{0.5pt}\end{center}

\hypertarget{ux989dux5916ux9650ux5236ux6761ux4ef6}{%
\subsection{额外限制条件}\label{ux989dux5916ux9650ux5236ux6761ux4ef6}}

\textasciitilde\textasciitilde 大部分资料显示,引物末端避免使用A或T可以有效避免扩增失败的情况,在此可以使用以下代码选择只以GC结尾的引物。

此外,有资料表明引物3端避开密码子的第三位也能提升扩增效率。值得注意的是,也有相关论点认为过多的GC在3端聚集可能会导致非特异性扩增。后续在单个引物的筛选阶段加入该限制可能也是有必要的。

该代码块默认不运行,需要时可选择单独运行。
\textasciitilde\textasciitilde{}

启用GC筛选(可选,默认关闭)

\begin{center}\rule{0.5\linewidth}{0.5pt}\end{center}

\hypertarget{ux5f15ux7269ux5bf9ux7684ux7b5bux9009}{%
\subsection{引物对的筛选}\label{ux5f15ux7269ux5bf9ux7684ux7b5bux9009}}

与单个引物筛选类似,此处使用的筛选条件有:\\
1. 扩增子长度\\
2. 引物对的Tm差异\\
3. 引物间的3端连续配对个数\\
4. 引物间任一连续碱基配对个数

在大部分资料中,限制引物间的配对主要是由认为选择的标准,如5bp或7bp。其本质在于考虑引物序列的互补链的相似性,可能后期可以将BLAST比对使用到其中,或者设计一个打分规则判断引物对的可用性。可能需要检测引物对的外部软件如IDT以获得高质量的结果。

调整扩增子长度、引物Tm差值、引物间3端配对上限(含)、引物间任意互补上限(含)后运行以下代码,符合条件的引物对(包括其序列、5-3端位置、GC含量、Tm值)将被放于数据框中并输出。

该步骤运算量较大,通常需要20s,但根据靶标序列的变大和限制条件的放松,运算时间可能会达数分钟。

\begin{Shaded}
\begin{Highlighting}[]
\CommentTok{\#引物对筛选}
\DocumentationTok{\#\#输入引物对限制条件}
\NormalTok{pcr\_max }\OtherTok{\textless{}{-}} \DecValTok{100}
\NormalTok{pcr\_min }\OtherTok{\textless{}{-}} \DecValTok{50}

\NormalTok{deltar\_Tm }\OtherTok{\textless{}{-}} \DecValTok{2}

\DocumentationTok{\#\#获取潜在引物对表,并根据限制筛选}

\NormalTok{get\_all\_primer\_sets }\OtherTok{\textless{}{-}} \ControlFlowTok{function}\NormalTok{(F\_primer,R\_primer,target\_size,pcr\_min,pcr\_max,deltar\_Tm,end\_match, total\_match)\{}
  
\NormalTok{  F\_primer\_list }\OtherTok{\textless{}{-}} \FunctionTok{split}\NormalTok{(F\_primer,F\_primer}\SpecialCharTok{$}\NormalTok{F\_accession)}

\NormalTok{  get\_primer\_sets }\OtherTok{\textless{}{-}} \ControlFlowTok{function}\NormalTok{(f\_primer\_data,r\_primer\_data, target\_size, pcr\_min, pcr\_max, deltar\_Tm) \{}
      
\NormalTok{      primer\_sets }\OtherTok{\textless{}{-}} \FunctionTok{merge}\NormalTok{(f\_primer\_data,r\_primer\_data, }\AttributeTok{all =}\NormalTok{ T)}
      
\NormalTok{      primer\_sets}\SpecialCharTok{$}\NormalTok{pcr\_length }\OtherTok{\textless{}{-}}\NormalTok{ target\_size}\SpecialCharTok{{-}}\NormalTok{(primer\_sets}\SpecialCharTok{$}\NormalTok{R\_start}\SpecialCharTok{+}\NormalTok{primer\_sets}\SpecialCharTok{$}\NormalTok{F\_start)}
\NormalTok{      primer\_sets}\SpecialCharTok{$}\NormalTok{pcr\_distance }\OtherTok{\textless{}{-}}\NormalTok{ target\_size }\SpecialCharTok{{-}}\NormalTok{(primer\_sets}\SpecialCharTok{$}\NormalTok{R\_end }\SpecialCharTok{+}\NormalTok{ primer\_sets}\SpecialCharTok{$}\NormalTok{F\_end)}
\NormalTok{      primer\_sets}\SpecialCharTok{$}\NormalTok{D\_Tm }\OtherTok{\textless{}{-}}\NormalTok{ primer\_sets}\SpecialCharTok{$}\NormalTok{R\_Tm }\SpecialCharTok{{-}}\NormalTok{ primer\_sets}\SpecialCharTok{$}\NormalTok{F\_Tm}
      
\NormalTok{      primer\_sets }\OtherTok{\textless{}{-}} \FunctionTok{subset}\NormalTok{(primer\_sets, }
                              \AttributeTok{subset =}\NormalTok{ pcr\_length }\SpecialCharTok{\textgreater{}=}\NormalTok{ pcr\_min}
                            \SpecialCharTok{\&}\NormalTok{ pcr\_length }\SpecialCharTok{\textless{}=}\NormalTok{ pcr\_max }
                            \SpecialCharTok{\&}\NormalTok{ pcr\_distance }\SpecialCharTok{\textgreater{}=} \DecValTok{0} 
                            \SpecialCharTok{\&}\NormalTok{ D\_Tm }\SpecialCharTok{\textless{}=}\NormalTok{ deltar\_Tm }
                            \SpecialCharTok{\&}\NormalTok{ D\_Tm }\SpecialCharTok{\textgreater{}=} \SpecialCharTok{{-}}\NormalTok{deltar\_Tm,}
                              \AttributeTok{select =} \FunctionTok{c}\NormalTok{(}\StringTok{"R\_seq"}\NormalTok{, }\StringTok{"R\_start"}\NormalTok{, }\StringTok{"R\_end"}\NormalTok{,}\StringTok{"R\_GC"}\NormalTok{,}\StringTok{"R\_Tm"}\NormalTok{, }\StringTok{"F\_seq"}\NormalTok{, }\StringTok{"F\_start"}\NormalTok{, }\StringTok{"F\_end"}\NormalTok{,}\StringTok{"F\_GC"}\NormalTok{,}\StringTok{"F\_Tm"}\NormalTok{,}\StringTok{"pcr\_length"}\NormalTok{))}
      
      \FunctionTok{return}\NormalTok{(primer\_sets)}
\NormalTok{  \}}
  
\NormalTok{  primer\_sets\_list }\OtherTok{\textless{}{-}}\FunctionTok{lapply}\NormalTok{(F\_primer\_list,get\_primer\_sets,}
                            \AttributeTok{f\_primer\_data =}\NormalTok{ R\_primer,}
                            \AttributeTok{target\_size =}\NormalTok{ target\_size,}
                            \AttributeTok{pcr\_min=}\NormalTok{pcr\_min,}
                            \AttributeTok{pcr\_max=}\NormalTok{pcr\_max,}
                            \AttributeTok{deltar\_Tm =}\NormalTok{ deltar\_Tm)}
  
\NormalTok{  primer\_sets }\OtherTok{\textless{}{-}} \FunctionTok{do.call}\NormalTok{(rbind,primer\_sets\_list)}
  
  \FunctionTok{return}\NormalTok{(primer\_sets)}
\NormalTok{\}}

\NormalTok{primer\_sets\_all }\OtherTok{\textless{}{-}} \FunctionTok{get\_all\_primer\_sets}\NormalTok{(}\AttributeTok{F\_primer=}\NormalTok{F\_primer,}\AttributeTok{R\_primer=}\NormalTok{R\_primer,}\AttributeTok{target\_size=}\NormalTok{target\_size,}\AttributeTok{pcr\_min=}\NormalTok{pcr\_min,}\AttributeTok{pcr\_max=}\NormalTok{pcr\_max,}\AttributeTok{deltar\_Tm =}\NormalTok{ deltar\_Tm)}

\DocumentationTok{\#\#根据引物间3端互补性筛选(可使用正则表达式优化)}
\NormalTok{end\_match\_length }\OtherTok{\textless{}{-}} \DecValTok{5}

\NormalTok{primer\_sets\_all}\SpecialCharTok{$}\NormalTok{R\_tail }\OtherTok{\textless{}{-}} \FunctionTok{str\_sub}\NormalTok{(primer\_sets\_all}\SpecialCharTok{$}\NormalTok{R\_seq, }\SpecialCharTok{{-}}\NormalTok{end\_match\_length)}
\NormalTok{primer\_sets\_all}\SpecialCharTok{$}\NormalTok{F\_tail }\OtherTok{\textless{}{-}} \FunctionTok{str\_sub}\NormalTok{(primer\_sets\_all}\SpecialCharTok{$}\NormalTok{F\_seq, }\SpecialCharTok{{-}}\NormalTok{end\_match\_length)}

\NormalTok{primer\_sets\_all}\SpecialCharTok{$}\NormalTok{total\_tail\_match }\OtherTok{\textless{}{-}} \FunctionTok{mapply}\NormalTok{(align\_all, }\FunctionTok{as.list}\NormalTok{(primer\_sets\_all}\SpecialCharTok{$}\NormalTok{F\_tail), }\FunctionTok{as.list}\NormalTok{(primer\_sets\_all}\SpecialCharTok{$}\NormalTok{R\_seq))}\SpecialCharTok{+}\FunctionTok{mapply}\NormalTok{(align\_all, }\FunctionTok{as.list}\NormalTok{(primer\_sets\_all}\SpecialCharTok{$}\NormalTok{R\_tail), }\FunctionTok{as.list}\NormalTok{(primer\_sets\_all}\SpecialCharTok{$}\NormalTok{F\_seq))}

\NormalTok{primer\_sets\_all }\OtherTok{\textless{}{-}} \FunctionTok{subset}\NormalTok{(primer\_sets\_all,}
                          \AttributeTok{subset =}\NormalTok{ total\_tail\_match }\SpecialCharTok{==} \DecValTok{0}\NormalTok{,}
                          \AttributeTok{select =} \FunctionTok{c}\NormalTok{(}\StringTok{"R\_seq"}\NormalTok{, }\StringTok{"R\_start"}\NormalTok{, }\StringTok{"R\_end"}\NormalTok{,}\StringTok{"R\_GC"}\NormalTok{,}\StringTok{"R\_Tm"}\NormalTok{, }\StringTok{"F\_seq"}\NormalTok{, }\StringTok{"F\_start"}\NormalTok{, }\StringTok{"F\_end"}\NormalTok{,}\StringTok{"F\_GC"}\NormalTok{,}\StringTok{"F\_Tm"}\NormalTok{,}\StringTok{"pcr\_length"}\NormalTok{))}

\DocumentationTok{\#\#根据引物间任意互补性筛选}
\NormalTok{total\_match\_length }\OtherTok{\textless{}{-}} \DecValTok{7}

\NormalTok{align\_between\_primers }\OtherTok{\textless{}{-}} \ControlFlowTok{function}\NormalTok{(q1,q2,length)\{}
  
\NormalTok{  select\_pattern }\OtherTok{\textless{}{-}} \FunctionTok{paste0}\NormalTok{(}\StringTok{".\{"}\NormalTok{,length,}\StringTok{"\}"}\NormalTok{)}
  
\NormalTok{  any\_match\_q1 }\OtherTok{\textless{}{-}} \FunctionTok{any}\NormalTok{(}\FunctionTok{str\_detect}\NormalTok{(q1,}\FunctionTok{str\_extract\_all}\NormalTok{(q2, }\AttributeTok{pattern =}\NormalTok{ select\_pattern)[[}\DecValTok{1}\NormalTok{]]))}
\NormalTok{  any\_match\_q2 }\OtherTok{\textless{}{-}} \FunctionTok{any}\NormalTok{(}\FunctionTok{str\_detect}\NormalTok{(q2,}\FunctionTok{str\_extract\_all}\NormalTok{(q1, }\AttributeTok{pattern =}\NormalTok{ select\_pattern)[[}\DecValTok{1}\NormalTok{]]))}
  \FunctionTok{return}\NormalTok{(any\_match\_q1 }\SpecialCharTok{==}\NormalTok{ T }\SpecialCharTok{|}\NormalTok{ any\_match\_q2}\SpecialCharTok{==}\NormalTok{ T)}
\NormalTok{\}}

\NormalTok{primer\_sets\_all}\SpecialCharTok{$}\NormalTok{any\_pair\_match }\OtherTok{\textless{}{-}} \FunctionTok{mapply}\NormalTok{(align\_between\_primers,}\FunctionTok{as.list}\NormalTok{(primer\_sets\_all}\SpecialCharTok{$}\NormalTok{R\_seq),}\FunctionTok{as.list}\NormalTok{(primer\_sets\_all}\SpecialCharTok{$}\NormalTok{F\_seq),}\AttributeTok{length =}\NormalTok{ total\_match\_length)}

\NormalTok{primer\_sets }\OtherTok{\textless{}{-}} \FunctionTok{subset}\NormalTok{(primer\_sets\_all, }
                      \AttributeTok{subset =}\NormalTok{ any\_pair\_match }\SpecialCharTok{==}\NormalTok{ F,}
                      \AttributeTok{select =} \FunctionTok{c}\NormalTok{(}\StringTok{"R\_seq"}\NormalTok{, }\StringTok{"R\_start"}\NormalTok{, }\StringTok{"R\_end"}\NormalTok{,}\StringTok{"R\_GC"}\NormalTok{,}\StringTok{"R\_Tm"}\NormalTok{, }\StringTok{"F\_seq"}\NormalTok{, }\StringTok{"F\_start"}\NormalTok{, }\StringTok{"F\_end"}\NormalTok{,}\StringTok{"F\_GC"}\NormalTok{,}\StringTok{"F\_Tm"}\NormalTok{,}\StringTok{"pcr\_length"}\NormalTok{))}

\FunctionTok{print}\NormalTok{(primer\_sets[}\FunctionTok{sample}\NormalTok{(}\FunctionTok{nrow}\NormalTok{(primer\_sets), }\DecValTok{15}\NormalTok{), ])}
\end{Highlighting}
\end{Shaded}

\begin{verbatim}
##                             R_seq R_start R_end  R_GC     R_Tm
## 7077.1356 TCCCCGATCCCCATCACGAATGG     256   278 60.87 59.67085
## 7523.195       CCCCGCGAAGGGGGTCAG    1665  1682 77.78 58.05688
## 9645.1621 CCGTTTCTCAGGCTCCCTCTCCG    1418  1440 65.22 60.22473
## 7774.1632 TCCCTCTCCGGAATCGAACCCTG    1431  1453 60.87 58.99876
## 3366.978   CCTGCCGGCGTAGGGTAGGCAC     299   320 72.73 62.92954
## 7071.433     CAACGGGTTACCCGCGCCTG     283   302 70.00 59.61596
## 7009.33        GCGTGCAGCCCCGGACAT     345   362 72.22 58.48298
## 6018.1583 TCGTCACTACCTCCCCGGGTCGG    1351  1373 69.57 62.63975
## 9548.1677 ACGTTCGAATGGGTCGTCGCCGC    1518  1540 65.22 62.99957
## 8531.756    CTCCTGGTGGTGCCCTTCCGT     646   666 66.67 59.76418
## 8934.1359 ATGGGGTTCAACGGGTTACCCGC     275   297 60.87 60.73604
## 4914.260      CCGGGCCGGGTGAGGTTTC     595   613 73.68 58.81625
## 10448.729   CAATCCTGTCCGTGTCCGGGC     580   600 66.67 59.05208
## 10753.438    CCTGCCGGCGTAGGGTAGGC     299   318 75.00 60.61440
## 3359.976   GTTCAACGGGTTACCCGCGCCT     280   301 63.64 60.91356
##                             F_seq F_start F_end  F_GC     F_Tm pcr_length
## 7077.1356   GCGCGCTACACTGACTGGCTC    1524  1544 66.67 59.35938         89
## 7523.195   TGGTCGCTCGCTCCTCTCCTAC     121   142 63.64 58.67256         83
## 9645.1621 GTGACCACGGGTGACGGGGAATC     395   417 65.22 60.93862         56
## 7774.1632  TGGTAGTCGCCGTGCCTACCAT     372   393 59.09 58.67256         66
## 3366.978      GGGGCTGCACGCGCGCTAC    1514  1532 78.95 62.46078         56
## 7071.433    CTGCACGCGCGCTACACTGAC    1518  1538 66.67 59.95653         68
## 7009.33     GTGGCGTTCAGCCACCCGAGA    1456  1476 66.67 60.33111         68
## 6018.1583   AGGAAGGCAGCAGGCGCGCAA     465   485 66.67 62.82820         53
## 9548.1677 GATAACCTCGGGCCGATCGCACG     298   320 65.22 61.10372         53
## 8531.756   GGGCAGCTTCCGGGAAACCAAA    1129  1150 59.09 59.19053         94
## 8934.1359  CACTGACTGGCTCAGCGTGTGC    1532  1553 63.64 59.71981         62
## 4914.260     GGCACCACCAGGAGTGGAGC    1211  1230 70.00 58.71482         63
## 10448.729 GAATTGACGGAAGGGCACCACCA    1198  1220 56.52 58.77405         91
## 10753.438 CTTAGATGTCCGGGGCTGCACGC    1503  1525 65.22 61.44611         67
## 3359.976      GATGTCCGGGGCTGCACGC    1507  1525 73.68 59.61379         82
\end{verbatim}

以上表格随机显示了结果中的15对可用引物。

\hypertarget{ux90e8ux5206ux7ed3ux679cux7684ux53efux89c6ux5316}{%
\subsection{部分结果的可视化}\label{ux90e8ux5206ux7ed3ux679cux7684ux53efux89c6ux5316}}

利用筛选过程的中间参数,可以得出在目标DNA上相应位置引物的GC含量与可用引物在靶标上的覆盖范围

\begin{Shaded}
\begin{Highlighting}[]
\FunctionTok{library}\NormalTok{(ggplot2)}

\NormalTok{CG\_overview }\OtherTok{\textless{}{-}} 
  \FunctionTok{ggplot}\NormalTok{(}\AttributeTok{data =}\NormalTok{ F\_primer\_all, }\FunctionTok{aes}\NormalTok{(}\AttributeTok{x=}\NormalTok{start, }\AttributeTok{y=}\NormalTok{GC))}\SpecialCharTok{+}
  \FunctionTok{geom\_line}\NormalTok{(}\AttributeTok{pch=}\DecValTok{2}\NormalTok{) }\SpecialCharTok{+}
  \FunctionTok{ggtitle}\NormalTok{(}\StringTok{"所选引物长度的GC含量与所在位置"}\NormalTok{)}\SpecialCharTok{+}
  \FunctionTok{labs}\NormalTok{(}\AttributeTok{x =} \StringTok{"位置"}\NormalTok{, }\AttributeTok{y =} \StringTok{"GC百分比"}\NormalTok{)}\SpecialCharTok{+}
  \FunctionTok{geom\_hline}\NormalTok{(}\FunctionTok{aes}\NormalTok{(}\AttributeTok{yintercept =}\NormalTok{ target\_GC), }\AttributeTok{colour =} \StringTok{"red"}\NormalTok{) }\SpecialCharTok{+}
  \FunctionTok{annotate}\NormalTok{(}\StringTok{\textquotesingle{}text\textquotesingle{}}\NormalTok{, }\AttributeTok{x =} \DecValTok{100}\NormalTok{, }\AttributeTok{y =}\NormalTok{ target\_GC}\DecValTok{{-}5}\NormalTok{, }\AttributeTok{label =} \FunctionTok{paste0}\NormalTok{(}\StringTok{"GC="}\NormalTok{,target\_GC,}\StringTok{"\%"}\NormalTok{), }\AttributeTok{colour =} \StringTok{"red"}\NormalTok{) }\SpecialCharTok{+}
  \FunctionTok{theme}\NormalTok{(}\AttributeTok{panel.grid =} \FunctionTok{element\_blank}\NormalTok{())}

\NormalTok{primer\_sets\_overview }\OtherTok{\textless{}{-}} 
  \FunctionTok{ggplot}\NormalTok{(}\AttributeTok{data =}\NormalTok{ primer\_sets, }\FunctionTok{aes}\NormalTok{(}\AttributeTok{x=}\NormalTok{R\_start, }\AttributeTok{y =}\DecValTok{1}\NormalTok{)) }\SpecialCharTok{+}
  \FunctionTok{geom\_segment}\NormalTok{(}\FunctionTok{aes}\NormalTok{(}\AttributeTok{x=}\DecValTok{0}\NormalTok{, }\AttributeTok{y=}\DecValTok{0}\NormalTok{, }\AttributeTok{xend =}\NormalTok{ target\_size, }\AttributeTok{yend =} \DecValTok{0}\NormalTok{))}\SpecialCharTok{+}
  \FunctionTok{geom\_segment}\NormalTok{(}\FunctionTok{aes}\NormalTok{(}\AttributeTok{x=}\NormalTok{target\_size}\SpecialCharTok{{-}}\NormalTok{R\_start, }\AttributeTok{y =} \FloatTok{0.5}\NormalTok{, }\AttributeTok{xend =}\NormalTok{ target\_size}\SpecialCharTok{{-}}\NormalTok{R\_end, }\AttributeTok{yend =} \FloatTok{0.5}\NormalTok{),}\AttributeTok{data =}\NormalTok{ primer\_sets, }\AttributeTok{color =} \StringTok{"red"}\NormalTok{) }\SpecialCharTok{+}
  \FunctionTok{geom\_segment}\NormalTok{(}\FunctionTok{aes}\NormalTok{(}\AttributeTok{x=}\NormalTok{F\_start, }\AttributeTok{y =} \DecValTok{1}\NormalTok{, }\AttributeTok{xend =}\NormalTok{ F\_end, }\AttributeTok{yend =} \DecValTok{1}\NormalTok{),}\AttributeTok{data =}\NormalTok{ primer\_sets, }\AttributeTok{color =} \StringTok{"blue"}\NormalTok{)}\SpecialCharTok{+}
  \FunctionTok{ggtitle}\NormalTok{(}\StringTok{"正反向可用引物在靶标上的位置"}\NormalTok{)}\SpecialCharTok{+}
  \FunctionTok{labs}\NormalTok{(}\AttributeTok{x =} \StringTok{"基因位置"}\NormalTok{,}\AttributeTok{y =} \StringTok{""}\NormalTok{)}\SpecialCharTok{+}
  \FunctionTok{coord\_fixed}\NormalTok{(}\AttributeTok{ratio =} \DecValTok{500}\NormalTok{)}\SpecialCharTok{+}
  \FunctionTok{theme}\NormalTok{(}\AttributeTok{axis.ticks =} \FunctionTok{element\_blank}\NormalTok{(),}\AttributeTok{axis.text.y =} \FunctionTok{element\_blank}\NormalTok{())}\SpecialCharTok{+}
  \FunctionTok{annotate}\NormalTok{(}\StringTok{"text"}\NormalTok{,}\AttributeTok{x=}\DecValTok{0}\NormalTok{, }\AttributeTok{y =} \DecValTok{1}\NormalTok{, }\AttributeTok{label =} \StringTok{"Forward"}\NormalTok{, }\AttributeTok{color =} \StringTok{"blue"}\NormalTok{)}\SpecialCharTok{+}
  \FunctionTok{annotate}\NormalTok{(}\StringTok{"text"}\NormalTok{,}\AttributeTok{x=}\DecValTok{0}\NormalTok{, }\AttributeTok{y =}\FloatTok{0.5}\NormalTok{,}\AttributeTok{label =} \StringTok{"Reverse"}\NormalTok{, }\AttributeTok{color =} \StringTok{"red"}\NormalTok{)}

\NormalTok{CG\_overview}
\end{Highlighting}
\end{Shaded}

\includegraphics{primer_setting_v.0.0.2_files/figure-latex/painting-1.pdf}

\begin{Shaded}
\begin{Highlighting}[]
\NormalTok{primer\_sets\_overview}
\end{Highlighting}
\end{Shaded}

\includegraphics{primer_setting_v.0.0.2_files/figure-latex/painting-2.pdf}

\hypertarget{ux8f93ux51faux7ed3ux679cux5230ux7ed3ux679cux76eeux5f55}{%
\subsection{输出结果到结果目录}\label{ux8f93ux51faux7ed3ux679cux5230ux7ed3ux679cux76eeux5f55}}

\begin{Shaded}
\begin{Highlighting}[]
\NormalTok{outpath }\OtherTok{\textless{}{-}} \FunctionTok{paste0}\NormalTok{(}\StringTok{"../results/"}\NormalTok{, }\FunctionTok{format}\NormalTok{(start\_time,}\StringTok{"\%y\%m\%d\_\%H\%M"}\NormalTok{))}

\FunctionTok{dir.create}\NormalTok{(outpath)}

\FunctionTok{write.csv}\NormalTok{(primer\_sets,}\AttributeTok{file =} \FunctionTok{paste0}\NormalTok{(outpath,}\StringTok{"/primer\_sets.csv"}\NormalTok{),}\AttributeTok{row.names =}\NormalTok{ F)}
\end{Highlighting}
\end{Shaded}

\hypertarget{ux603bux7ed3}{%
\subsection{总结}\label{ux603bux7ed3}}

\begin{Shaded}
\begin{Highlighting}[]
\NormalTok{end\_time }\OtherTok{\textless{}{-}} \FunctionTok{Sys.time}\NormalTok{()}

\NormalTok{message }\OtherTok{\textless{}{-}} \FunctionTok{paste0}\NormalTok{( }\StringTok{"本次运行耗费时间共"}\NormalTok{,}\FunctionTok{round}\NormalTok{(end\_time}\SpecialCharTok{{-}}\NormalTok{start\_time,}\DecValTok{3}\NormalTok{),}\StringTok{"}\SpecialCharTok{\textbackslash{}n}\StringTok{ 设置的引物长度于"}\NormalTok{,primer\_size\_min,}\StringTok{"与"}\NormalTok{,primer\_size\_max,}\StringTok{"之间;}\SpecialCharTok{\textbackslash{}n}\StringTok{ Tm值于"}\NormalTok{,Tm\_min,}\StringTok{"与"}\NormalTok{,Tm\_max,}\StringTok{"之间;}\SpecialCharTok{\textbackslash{}n}\StringTok{ 最终的扩增子长度在"}\NormalTok{,pcr\_min,}\StringTok{"到"}\NormalTok{,pcr\_max,}\StringTok{"。}\SpecialCharTok{\textbackslash{}n}\StringTok{ 共有"}\NormalTok{,}\FunctionTok{length}\NormalTok{(primer\_sets}\SpecialCharTok{$}\NormalTok{R\_seq),}\StringTok{"对引物被筛选出来"}\NormalTok{)}

\FunctionTok{cat}\NormalTok{(message)}
\end{Highlighting}
\end{Shaded}

\begin{verbatim}
## 本次运行耗费时间共37.202
##  设置的引物长度于18与23之间;
##  Tm值于58与63之间;
##  最终的扩增子长度在50到100。
##  共有43025对引物被筛选出来
\end{verbatim}

\begin{Shaded}
\begin{Highlighting}[]
\FunctionTok{write}\NormalTok{(message, }\AttributeTok{file =} \FunctionTok{paste0}\NormalTok{(outpath,}\StringTok{"/summary\_"}\NormalTok{,}\FunctionTok{basename}\NormalTok{(input\_path)))}
\end{Highlighting}
\end{Shaded}


\end{document}
